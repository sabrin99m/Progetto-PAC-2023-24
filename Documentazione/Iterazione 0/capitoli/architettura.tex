\section{Architettura}

L'architettura del sistema è stata definita attraverso il deployment diagram mostrato in figura \ref{fig: DeployementFreeStyle}, espresso in notazione free style.
Si identifica un pattern architetturale three-tier, dove:
\begin{enumerate}
\item Il Presentation Layer è costituito dalle applicazioni lato client che vengono utilizzate dagli utenti e hanno il solo compito di recuperare le informazioni dal sistema e visualizzarle;
\item L'Application Layer gestisce la logica di business. È costituito da due macro-sistemi specializzati:
\begin{itemize}
    \item \textbf{Gestore profilo}, responsabile di tutto ciò che riguarda i profili degli utenti, come gestire la registrazione, il login
    e soddisfare le query inerenti all'utente. Questo gestore si connette ai servizi di Firebase per gestire l'autenticazione in fase di login;
    \item \textbf{Gestore escursione}, responsabile di gestire le richieste inerenti alle escursioni, come ottenere una lista o i dettagli di uno specifico
    evento. Per fornire informazioni meteorologiche riguardo agli eventi, il gestore si connette ai servizi di Visual Crossing tramite API call. La risposta viene elaborata
    in loco e restituita al client in un formato più breve da definirsi in futuro;
\end{itemize}
\item Il Data Layer si occupa della persistenza dei dati ed è costituito da un database dove sono conservate le informazioni di ogni evento e utente registrato alla 
piattaforma.
\end{enumerate}

\begin{figure}[ht!]
    \centering
    \begin{subfigure}{0.9\textwidth}
        \includegraphics[width=\linewidth]{immagini/DeployemenFreeStyle.png}
    \end{subfigure}
    \caption{Deployment Diagram in versione free style}
    \label{fig: DeployementFreeStyle}
\end{figure}
